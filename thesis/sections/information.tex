%% LaTeX2e class for student theses
%% sections/content.tex
%% 
%% Karlsruhe Institute of Technology
%% Institute for Program Structures and Data Organization
%% Chair for Software Design and Quality (SDQ)
%%
%% Dr.-Ing. Erik Burger
%% burger@kit.edu
%%
%% Version 1.3.5, 2020-06-26

\chapter{Information to sort}
\label{ch:Info}

\section{MARL - A Comprehensive Survey of Multiagent Reinforcement Learning - 2008}
\href{https://ieeexplore.ieee.org/abstract/document/4445757}{MARL - A Comprehensive Survey of Multiagent Reinforcement Learning - 2008}
\\
\underline{Benefits}
\begin{itemize}[noitemsep,nolistsep]
	\item can be parallelized.
	\item can use experience sharing via communication, or with a teacher-learner relationship.
	\item Failure of one agent can be covered by other agents.
	\item insertion of new agents => scaleable.
	\item MARL Complexity: Exponential in number of agents.
	\item exploration (new knowledge) - exploitation (current knowledge) - Tradeoff.
	\item They explore about the environment and other agents.
	\item need for coordination.
\end{itemize}
\underline{Application Domains}
\begin{itemize}[noitemsep,nolistsep]
	\item simulation better than real-life (better scalability and robustness).
	\item Distributed Control: for controlling processes (for larger industry plants).
	\begin{itemize}[noitemsep,nolistsep]
		\item avenue for future work.
		\item used for traffic, power or sensory networks.
		\item could also be used for pendulum systems.
	\end{itemize}
	\item Robotic Teams (Multirobot):
	\begin{itemize}[noitemsep,nolistsep]
		\item simulated 2D space. 
		\item navigation: Reach a goal with obstacles. Area sweeping (retrival of objects (also cooperative)).
		\item pursuit: Capture a prey robot.
	\end{itemize}
	\item Automated Trading: Exchange goods on electronic markets with negotiation and auctions.
	\item Resource Management: Cooperatie team to manage resources or as clients. (routing, load balancing).
\end{itemize}
\underline{Practicallity and Future works}
\begin{itemize}[noitemsep,nolistsep]
	\item Scalability Problem: Q-functions do not scale well with the size of the state-action space.
	\begin{itemize}[noitemsep,nolistsep]
		\item Approximation needed: for discrete large state-action spaces, for continous states and discrete actions or continious state and action.
		\item Heuristic in nature and only work in a narrow set of problems.
		\item Could use theoretical results on single-agent approximate RL.
		\item also could use discovery and exploitation of the decentralized, modular structure of the multiagent task.
	\end{itemize}
	\item MARL without prior knowledge is very slow.
	\begin{itemize}[noitemsep,nolistsep]
		\item Need humans to teach the agent.
		\item shaping: first simple task then scale them.
		\item could use reflex behavior.
		\item Knowledge about the task structure.
	\end{itemize}
	\item Incomplete, uncertain state measurements could be handled with partiall observability techniques (Markov decision process).
	\item Multiagent Goals needs a stable learning process for the environment and an adaption for the dynamics of other agents.
	\item using game-theory-based analysis to apply to the dynamics of the environment.
\end{itemize}

\section{!MAS - An Introduction to Multi-Agent Systems - 2010}
\href{https://link.springer.com/chapter/10.1007/978-3-642-14435-6_1}{MAS - An Introduction to Multi-Agent Systems - 2010}
\\
\underline{Benefits of using MAS in large systems}
\begin{itemize}[noitemsep,nolistsep]
	\item Increase in the speed and efficiency of the operation due to parallel computation and asynchronous operation.
	\item Graceful degradation whone one or more of the agent fail, thus increasing realibility and robustness of the system.
	\item Scalability and flexibility - Agents can be added as and when necessary.
	\item Cost Reduction: Individual agents cost much less than a centralized architecture
	\item Reusability: Agents with a modular structure can be easily replaced in other systems or be upgraded more easily than a monolithic sysetm.
\end{itemize}
\underline{Challenges of using MAS in large systems}
\begin{itemize}[noitemsep,nolistsep]
	\item environment: An agents action modify its own environment but also that of its neighbours. therefore they need to predict the action of the other agents so that they can reach a goal. This can be an unstable system. Environment dynamic: Is the effect caused by other agents or by the variation in the environment?
	\item perception: limited sensing range => each agent only has partial observability for the environment. Therefore the decisions reached might be sub-optimal.
	\item Abstraction: ???
	\item conflict resolution: lack of global view => conflict. therefore information on constraints, action preferences and goal prioritoes must be shared between agents. When to communicate what to which agent?
	\item Inference: Single-Agent: State-Action-Space can be mapped with trial and error. Multi-agent: each agent may or may not interact with each other. If they are heterogenous, they might even compete and have different goals. You need a fitting inference machanism
\end{itemize}
\subsection{Classification of MAS}
\underline{Internal Architecture}
\begin{itemize}[noitemsep,nolistsep]
	\item homogeneous: all agents have the same internal architecture (Local Goals, Sensor Capabilities, Internal states, Inference Mechanism and Possible Actions). In a typical distributed environment, overlap of sensory inputs is rarely present
	\item Heterogeneous: agents may differ in ability, structure and functionality. Because of the dynamics and location the actions chosen might differ between agents. their local goals may contradict the objective of other agents.
\end{itemize}
\underline{Agent Organization}
\begin{itemize}[noitemsep,nolistsep]
	\item hierarchical: typical: tree-structure. At different heights, different levels of autonomy. data from lower levels flow upwards. Control signal flows from high to low in the hierarchy.
	\begin{itemize}[noitemsep,nolistsep]
		\item simple: the decision making authority is a single agent of highest level. BUT: single point of Failure
		\item uniform: authority is distributed among the various agents, for better efficiency, fault tolerance, graceful degradation. Decisions made by agent with appropriate information. (MAS - TrafficControl - Neural Networks for Continuous Online Learning and Control - 2006)
	\end{itemize}
	\item holonic: fractal structure of several holons. Self-repeating. Used for large organizational behaviours in manufacturing and business. 
	\begin{itemize}[noitemsep,nolistsep]
		\item An agent that appears as a single entity might be composed of many sub-agents. They are not predetermined, but form through commitments.
		\item Each holon has a head agent that communicates with the environment or with other agents in the environment. It is selected either randomly, through a rotation policy, or selected by resource availability, communicaton capability.
		\item Holons can be nested to form Superholons.
		\item compare to tree: in Holons cross tree interactions and overlapping of holons is allowed.
		\item pro: abstraction good degree of freedom, good agent autonomy.
		\item contra: abstraction makes it difficult for other agents to predict the resulting actions of the holon.
	\end{itemize}
	\item coalitions: group of agents come together for a short time to increase utility or performance of the individual agents in a group. they cease to exist when the performance goal is achieved.
	\begin{itemize}[noitemsep,nolistsep]
		\item coalition may have either a flat or a hierarchical architecture.
		\item It may have an leading agent to act as a representative. 
		\item overlap is allowed. this increased complexity of computation of the negotiation strategy.
		\item You can have one coaltion with all agents => maximum performance of system. Impractical due to restraints on communication and resources.
		\item minimize amount of colations: because of the cost of creating and dissolving a colation group.
	\end{itemize}
	\item teams: agents work together to increase the overall performance of the group, rather than working as individual agents.
	\begin{itemize}[noitemsep,nolistsep]
		\item their interactions can be arbitrary and the goals and roals can vary with the performance of the group.
		\item large team size is not beneficial under all conditions. some compromises must be made.
		\item large teams offer a better visibility of the environment. but is slower computation wise. Learning-Performance Tradeoff.
		\item computation cost usually much greater than coalitions.
	\end{itemize}
\end{itemize}
\underline{Communication}
\begin{itemize}[noitemsep,nolistsep]
	\item local communication: agents directly communicate similar to message passing. there is no place to store information. creates distributed architecture. used in: (25),(37),(38).
	\item blackboards: a group of agents share a data repository which is provided for efficient storage.
	\begin{itemize}[noitemsep,nolistsep]
		\item can hold design data and control knowledge, accessable by the agents.
		\item control shell: notfies the agent when relevant data is available.
		\item single point of failure.
	\end{itemize}
	\item agent communication language (ACL): common framework for interaction and information sharing. (40).
	\begin{itemize}[noitemsep,nolistsep]
		\item procedural approach: modelled as a sharing of the precedural directives. Shared how an agent does a specific task or the entire working of the agent itself. Script Languages often used. Disadvantage: necessitiy of providing information on the recipient agent, which is in most cases partially known. Also how to merge the scripts into one executable. Not preferred method.
		\item declarative approach: sharing of statements for definitions, assumptions assertions, axioms etc. Short declarative statements as length increases probability of information corruption. Example: ARPA knowledge sharing effort.
		\item Best known inner languages: Knowledge Interchange Format. Information exchange is implicitly embedded in KIF. But the package size grows with the increase in embedded information. Solution: High-level Languages like KQML (Knowledge QUery and Manipulation Language)
	\end{itemize}
\end{itemize}
\underline{Decision making in Multi-Agent Systems}
\begin{itemize}[noitemsep,nolistsep]
	\item undercainty: effects of a specific actions on the environment and dynamics because of the other agents.
	\item Methodology to try and find a joint action or equilibrium point which maximizes the reward of every agent.
	\item Typically modelled with game theory method. Strategic games:
	\begin{itemize}[noitemsep,nolistsep]
		\item a set of players (agents)
		\item Foreach player, there is a set of actions
		\item Foreach player, the prefeernces over a set of actions profiles
		\item payoff with the combination of action, a joint-action, that is assumed to be predefined.
		\item all actions are observable forall agents.
		\item make the assumption that all participating agents are rational.
	\end{itemize}
	\item Nash equilibrium: for a payoff matrix: An action profile (joint-action), where no player can do better by choosing one of the actions differently, given that the other player chose a specific action.
	\item there might be multiple nash equilibrium, so that there is no dominant solution. Here the coordination of MAS is needed to find a solution.
	\item Iterated Elimination Method: Strongly dominated actions are iteratively eliminated. This fails if there are no strictly dominated actions available.
\end{itemize}
\underline{Coordination}
\begin{itemize}[noitemsep,nolistsep]
	\item agents work in parallel, therefore they need to be coordinated or synchronize the actions to ensure stability of the system.
	\item other reasons: prevent chaos, meet global constraints, utilize distributed resources, prevent conflicts, improve efficiency.
	\item achievable with constraints on the joint actions or by using informatil collated from neighbouring agents. Used to find the equilibrium action.
	\item payoff matrix necessary might be difficult to determine. It increases expenentially in the number of agents and action choices.
	\item dividing the game into subgames: roles (permitted actions is reduced, good for distributed coordination or centralized coordination)
	\item Coordination via Protocol.
	\begin{itemize}[noitemsep,nolistsep]
		\item negotioation to arrive an approdiate solutions.
		\item Agents assume the role of manager (divide the problem) and contractor (who deals with the subproblems).
		\item The manager and contractor are working in a bidding system.
		\item Example: FIPA model
		\item disadvantage: assumption of the existence of an cooperative agent. It is very communication intensive
	\end{itemize}
	\item Coordination via Graphs: Problem is subdivided into easer problems. Assume the payoffs can be linear combinated from the local payoffs of the sub-games. THen just eliminate agents to find the optimal joint.
	\item Can also use belif models. Internal models of an agent on how he believes the environment works (needs to differentiate between environment and effects of other agents).
\end{itemize}
\underline{Learning}
\begin{itemize}[noitemsep,nolistsep]
	\item active learning: analysing the observations to creat a belief or internal model of the corresponding situated agent's environment. 
	\begin{itemize}[noitemsep,nolistsep]
		\item can be performed by using a deductive, inductive or probabilistic reasoning approach.
		\item deductive: inference to explain an instance or state-action sequence using his knowledge. It is deduced or inferred from the original knowledge it is nothing new. It could form new parts of the knowledge base. uncertainty is usually disregarded (not good for real-time)
		\item inductive: learning from observations of state-action pair. Good when environment can be presented in terms of some generalized statements. they use the correlation between observations and the action space.
		\item probabilistic: assumption: knowledge base or belief model can be represented as probabilities of occurrence of events. observations of the environment is used to predict the internal state of the agent. Good example: Bayesian learning. Difficult for MAS, as the joint probability scales poorly in the number of agents.
	\end{itemize}
	\item reactive learning: updating belief without having the actual knowledge of what needs to be learnt.
	\begin{itemize}[noitemsep,nolistsep]
		\item useful when the underlying model of the agent or the environment is not known clearly and are black boxes.
		\item can be ssen in agents which utilize connections systems such as NN.
		\item can use reactive multi-agent feed forward neural networks.
		\item they depend on the application domain and are therefore rarely employed in real world scenarios.
	\end{itemize}
	\item learning based on consequences:
	\begin{itemize}[noitemsep,nolistsep]
		\item learning methods based on evaluation of the goodness of selected action. like in reinforcement learning.
		\item programming the agents using reward and punishment scalar signals without specifying how the task is to be achieved.
		\item learnt through trial and error and interaction with the environment.
		\item usually used when action space is small and descret. Recent developments allow them to work in continious and large state-action space scenarios.
		\item An agent is usally represented as a Markov Decision Process.
		\item Expectaation operator optmal policy is the argmax of the Q-value, which uses the bellman equation. Bellman equation is solved iteratively.
		\item The solution is referred to as q-learning method.
		\item For MAS the reinforcement learning method has the problem of combinatorial explosion in the state-action pairs.
		\item The information must be passed between the agents for effective learning.
	\end{itemize}
\end{itemize}

\section{Artificial Intelligence - A modern Approach}
\subsection{Agents and Environments}
p.34
\begin{itemize}[noitemsep,nolistsep]
	\item \textbf{agent}: anything that perceives its \textbf{environment} through \textbf{sensors} and acting upon that environment using \textbf{actuators}.
	\item \textbf{percept}: agent’s perceptual inputs at any given instance. Percept sequence is a complete history of perception.
	\item agents choice of action decided upon the history of perception, but not anything it has not perceived.
	\item its behavior is described by the \textbf{agent function}, which is internally implemented by the \textbf{agent program}.
\end{itemize}

\subsection{Rational Agent}
p.36
\begin{itemize}[noitemsep,nolistsep]
	\item \textbf{rational agent}: it does the correct thing. Correctness is determined by a performance measure, which is determined by the changed environment states.
	\item design \textbf{performance measures} according to what one actually wants in the environment, rather than according to how one thinks the agent should behave.
	\item rational depends on:
	\begin{itemize}
		\item the performance measure that defines the criterion of success
		\item the agent’s prior knowledge of the environment.
		\item The actions that the agent can perform.
		\item The agent’s percept sequence of data.
	\end{itemize}
	\item depending on the measures the agent might be rational or not. 
	\item an \textbf{omniscient agent} knows the actual outcome of its actions and can act accordingly, but this is impossible in reality.
	\item rationality maximizes expected performance, while perfection (omniscient) maximizes actual performance.
	\item agents can do actions in order to modify future percepts, called \textbf{information gathering, or exploration}.
	\item rational agents learn as much as possible from what it perceives.
	\item his knowledge can be augmented and modified as it gains experience.
	\item if the agent relies on the prior knowledge of its designer rather than on its own percepts, we say that the agent lacks \textbf{autonomy}.
	\item it should learn what it can to compensate for partial or incorrect prior knowledge.
	\item give it some initial knowledge and the ability to learn, so it will become independent of its prior knowledge.
\end{itemize}

\subsection{Nature of Environments}
p.40
\begin{itemize}[noitemsep,nolistsep]
	\item \textbf{task environments}: the “problems” to which rational agents are the “solutions”.
	\item Describe the task environment in the following aspects P(Performance measure), E(Environment), A(Actuators), S(Sensors).
	\item \textbf{fully observable}: the agent’s sensors give it access to the complete state of the environment. All aspects that are relevant to the choice of actions
	\item \textbf{partially observable}: otherwise. Because of missing sensors or noise.
	\item no sensors: unobservable
	\item single-agent environments and multi-agent environments.
	\item multi-agent can be either competitive (chess) or cooperative (avoiding collisions maximizes performance).
	\item \textbf{communication} emerges as a rational behavior in multiagent environments.
	\item randomized behavior is rational because it avoids the pitfalls of predictability.
	\item \textbf{Deterministic}: next state of environment is completely determined by the current state and the action executed by the agent, otherwise it is \textbf{stochastic}.
	\item you can ignore uncertainty that arises purely from the actions of other agents in a multiagent environment.
	\item If the environment is partial observable, it could appear to be stochastic, which implies quantifiable outcomes in terms of probabilities.
	\item an environment is \textbf{uncertain} if it is not fully observable or not deterministic. 
	\item \textbf{episodic}: the agent’s experience is divided into atomic episodes. In each the agent receives a percept and performs a single episode. The next episode does not depend on the actions taken in previous episodes, otherwise it is \textbf{sequential}.
	\item When the environment can change while the agent is deliberating, then the environment is \textbf{dynamic} for that agent otherwise it is \textbf{static}.
	\item if the environment itself does not change with the passage of time but the agent’s performance score does, then we say the environment is \textbf{semi dynamic}.
	\item \textbf{discrete/continuous} applies to the state of the environment, to the way time is handled, and to the percepts and actions of the agents.
	\item \textbf{known vs. unknown}: refers to the agent’s state of knowledge about the “laws of physics” of the environment. Known environment, the outcomes for all actions are given, otherwise the agent needs to learn how it works. An environment can be known, but partially observable (solitaire: I know the rules but still unable to see the cards that have not yet been turned over)
	\item hardest case: partially observable, multiagent, stochastic, sequential, dynamic, continuous, and unknown
	\item \textbf{environment class}: multiple environment scenarios to train it for multiple situations.
	\item you can create an \textbf{environment generator}, that selects environments in which to run the agent.
\end{itemize}

\subsection{Structure of Agents}
p.46
\begin{itemize}[noitemsep,nolistsep]
	\item agent = architecture (computing device) + program (agent program).
	\item agent programs take the current percept as input and return an action to the actuators.
	\item agent program takes the current percept, agent function which takes the entire percept history.
	\item \textbf{table driven agent}: Uses a table of actions indexed by percept sequences. This table grows way to fast and is therefore not practical.
\end{itemize}

\underline{Simple reflex agents:}
\begin{itemize}[noitemsep,nolistsep]
	\item \textbf{simple reflex agents}: Select the actions on the basis of the current percept, ignoring the rest of the history.
	\item \textbf{condition-action-rule}: these agents create actions in a specific condition (if-then). These connections can be seen as reflexes.
	\item uses an \textbf{interpret-input} function as well as a \textbf{rule-match} function.
	\item they need the environment to be fully observable. They could run into infinite loops.
	\item you can mitigate this by using randomization for the actions. Which is non-rational for single agent environments.
\end{itemize}

\underline{Model-based reflex agents:}
\begin{itemize}[noitemsep,nolistsep]
	\item keep track of the part of the world an agent cannot see now. It maintains some sort of \textbf{internal state} that depends on the percept history.
	\item agents needs to know how the world evolves independently of the agent and how the agent’s own actions affect the world.
	\item with this it creates a \textbf{model} of the world hence it is called model-based agent.
	\item it needs to update this state given sensor data.
	\item this model is a \textbf{best guess} and does not determine the entire current state of the environment exactly.
\end{itemize}

\underline{Goal-based agents:}
\begin{itemize}[noitemsep,nolistsep]
	\item an agent needs some sort of \textbf{goal information} that describes situations that are desirable. This can also be combined with the model.
	\item Usually agents need to do multiple actions to fulfill a goal which requires \textbf{search} and \textbf{planning}.
	\item this also involves consideration of the future.
	\item the goal-based agent’s behavior can be easily changed to go to a different destination by using a goal where a reflex agent needs completely now rules.
\end{itemize}

\underline{Utility-based agents:}
\begin{itemize}[noitemsep,nolistsep]
	\item goals provide a crude binary distinction between good and bad states.
	\item use an internal \textbf{utility function} to create a performance measure.
	\item if the external performance measure and the internal utility function agree, the agent will act rationally.
	\item if you have conflicting goals the utility function can specify the appropriate \textbf{tradeoff}.
	\item if multiple goals cannot be achieved with certainty, utility provides a way to determine the \textbf{likelihood} of success.
	\item a rational utility-based agent chooses the action that \textbf{maximizes the expected utility}.
	\item any rational agent must behave as if it possesses a utility function whose expected value it tries to maximize.
	\item a utility-based agent must model and keep track of its environment.
\end{itemize}

\underline{Learning Agents:}
\begin{itemize}[noitemsep,nolistsep]
	\item it allows the agent to operate in initially unknown environments and to become more competent than its initial knowledge alone might allow.
	\item 4 conceptual components: \textbf{learning element} (responsible for improvements), \textbf{performance element} (select external action), \textbf{critic} (gives feedback to change the learning element), \textbf{problem generator} (suggesting actions that lead to new and informative experiences).
	\item critic tells the learning element how well the agent is doing given a performance standard. It tells the agent which percepts are good and which are bad.
	\item problem generator allows for exploration and suboptimal actions to discover better actions in the long run.
	\item learning element: simplest case: learning directly from the percept sequence.
	\item the \textbf{performance standard} distinguishes part of the incoming percept as a reward or penalty that provides direct feedback on the quality of the agent’s behavior.
\end{itemize}

\underline{How the components of agent programs work:}
\begin{itemize}[noitemsep,nolistsep]
	\item \textbf{atomic representation}: Each state of the world is indivisible. Algorithms like search and game-playing, Hidden Markov models and Markov decision models work like this.
	\item \textbf{factored representation}: splits up each state of a fixed set of variables or attributes which each can have a value. Used in constraint satisfaction algorithms, propositional logic, planning, Bayesian networks.
	\item \textbf{structured representation}: here the different states have connections to each other. Used in relational databases, first-order logic, first-order probability models, knowledge-based learning and natural language understanding.
	\item more complex representations are more \textbf{expressive} and can capture everything more concise.
\end{itemize}

\subsection{Multiagent Planning}
p.425
\begin{itemize}[noitemsep,nolistsep]
	\item each agent tries to achieve is own goals with the help or hindrance of others
	\item wide degree of problems with various degrees of \textbf{decomposition of the monolithic agent}.
	\item multiple concurrent effectors => \textbf{multieffector planning} (like type and speaking at the same time).
	\item effectors are physically decoupled => \textbf{multibody planning}.
	\item if relevant sensor information foreach body can be pooled centrally or in each body ~ like single-agent problem.
	\item When communication constraint does not allow that: \textbf{decentralized planning problem}. planning phase is centralized, but execution phase is at least partially decoupled.
	\item single entity is doing the planning: one goal, that every body shares.
	\item When bodies do their own planning, they may share identical goals.
	\item \textbf{multibody}: centralized planning and execution send to each.
	\item \textbf{multiagent}: decentralized local planning, with coordination needed so they do not do the same thing.
	\item Usage of \textbf{incentives} (like salaries) so that goals of the central-planner and the individual align.
\end{itemize}

\underline{Multiple simultaneous actions:}
\begin{itemize}[noitemsep,nolistsep]
	\item \textbf{correct plan}: if executed by the actors, achieves the goal. Though multiagent might not agree to execute any particular plan.
	\item \textbf{joint action}: An Action for each actor defined => joint planning problem with branching factor b\^n (b = number of choices).
	\item if the actors are \textbf{loosely coupled} you can describe the system so that the problem complexity only scales linearly.
	\item standard approach: pretend the problems are completely decoupled and then fix up the interactions.
	\item \textbf{concurrent action list}: which actions must or most not be executed concurrently. (only one at a time)
\end{itemize}


\underline{Multiple agents: cooperation and coordination }
\begin{itemize}[noitemsep,nolistsep]
	\item each agent makes its own plan. Assume goals and knowledge base are shared.
	\item They \textbf{might choose different plans} and therefore collectively not achieve the common goal.
	\item \textbf{convention}: A constraint on the selection of joint plans. (cars: do not collide is achieved by “stay on the right side of the road”).
	\item widespread conventions: social laws.
	\item absence of convention: use communication to achieve common knowledge of a feasible joint plan.
	\item The agents can try to \textbf{recognize the plan other agents want to execute} and therefore use plan recognition to find the correct plan. This only works if it is unambiguously.
	\item an \textbf{ant} chooses its role according to the local conditions it observes.
	\item ants have a convention on the importance of roles.
	\item ants have some learning mechanism: a colony learns to make more successful and prudent actions over the course of its decades-long life, even though individual ants live only about a year.
	\item Another Example: \textbf{Boid}
	\item If all the boids execute their policies, the flock inhibits the emergent behavior of flying as a pseudorigid body with roughly constant density that does not disperse over time.
	\item \textbf{most difficult multiagent} problems involve both cooperation with members of one’s own team and competition against members of opposing teams, all without centralized control.
\end{itemize}

\subsection{Game Theory}
p.666

\subsection{Mechanism Design for Multiple Agents}
p.679

\subsection{Adversarial Search}
p.182

\subsection{Probabilistic Reasoning over Time}
p.587

\subsection{Reinforcement Learning}
p.830

\subsection{Planning Uncertain Movements (Potential Fields)}
p.993


\section{Ant Colony Optimization}
\subsection{Wikipedia Article}
\href{https://en.wikipedia.org/wiki/Ant_colony_optimization_algorithms}{Ant Colony Optimization Algorithm, Wikipedia}
\begin{itemize}[noitemsep,nolistsep]
	\item is used for solving computational problems which can be reduced to finding good paths through graphs.
	\item artificial ants locate optimal soluions by moving through a parameter space represneting all possible solutions.
	\item they record their positions and the quality of their solutions for later iterations to find better solutions (pheromones).
\end{itemize}
\section{Reinforcement Learning}
\subsection{Algorithmia Blog}
\href{https://algorithmia.com/blog/introduction-to-reinforcement-learning}{Introduction to Reinforcement Learning}
\begin{itemize}[noitemsep,nolistsep]
	\item \textbf{Policy Learning}: Policy is a function: (state) -> (action). (if you approach an enemy and the enemy is stronger than you, turn backwards).
	\item Can use Neural Nets to approximate complicated functions
	\item \textbf{Q-Learning / Value Functions}: (state, action) -> (value). It also adds in all of the potential future values that this action might bring you.
	\item Approximate Q-Learning Functions with Neural Nets: DQN (RL - DQN - Human-level control through deep reinforcement - 2015)
	\item Newer way to approximate Q-Functions: A3C ( \href{https://medium.com/emergent-future/simple-reinforcement-learning-with-tensorflow-part-8-asynchronous-actor-critic-agents-a3c-c88f72a5e9f2}{Tutorial}, RL - A3C - Asynchronous Methods for Deep Reinforcement Learning - 2016)
	\item \textbf{Challenges}:
	\begin{itemize}[noitemsep,nolistsep]
		\item Reinforcement Learning requires a ton of training data, that other algorithms can get to more efficiently.
		\item RL is a general algorithm. If the problem has a domain-specific solution that might work better than RL. Tradeoff between scope and intensity.
		\item Most pressing Issue: Design of the reward function. it could get stuck in local optima
	\end{itemize} 
\end{itemize}
\subsection{Freecodecamp}
\href{https://www.freecodecamp.org/news/an-introduction-to-reinforcement-learning-4339519de419/}{An introduction to Reinforcement Learning}
\begin{itemize}[noitemsep,nolistsep]
	\item $State\ S_t, Reward\ R_t, Action\ A_t$
	\item \textbf{Reward Hypothesis}: All goals can be described by the maximization of the expected cumulative reward: $G_t = \sum_{k=0}^T R_{t+k+1}$
	\item But as earlier rewards are more probable to happen you need to increase their perceived value. Therefore you need a factor $0 \leq \gamma < 1$.
	\item Large $\gamma$, Agent cares about long-term reward. Small $\gamma$, Agent cares more about short term reward.
	\item \textbf{Discounted Accumulative Rewards}: $G_t = \sum_{k=0}^\infty \gamma^k R_{t+k+1},\ where\ \gamma \in [0,1)$ 
	\item \textbf{Episodic tasks}: starting point and an ending point (terminal state), this creates an episode.
	\item \textbf{Continuous Tasks}: Tasks that continue forever (no terminal state).
	\item Learning Methods: Collecting the rewards at the end of the episode for the feature (Monte-Carlo), or Estimate the rewards at each step (Temporal Difference Learning)
	\item \textbf{Monte-Carlo}: $V(S_t) \leftarrow V(S_t) + \alpha [G_t - V(S_t)]$. Left-Side: $V(S_t)$ Maximum expected Future, Right-Side: $V(S_t)$ Former estimation of maximum expected future. $\alpha$: learning rate.
	\item \textbf{TD-Learning}: $V(S_t) \leftarrow V(S_t) + \alpha [R_{t+1} + \gamma V(S_{t+1}) - V(S_t)]$. $R_{t+1} + \gamma V(S_{t+1})$ is the TD-Target. TD-Target is an estimation, by updating it via a one-step target.
	\item \textbf{Exploration/Exploitation Tradeoff}: Exploration (finding more information about the environment), Exploitation (using known information to maximize the reward). The Agent might find better rewards by doing exploration.
	\item \textbf{Value Based RL}: OPtimize the value function V(s), that tells us the maximum expected future reward.
	\begin{itemize}[noitemsep,nolistsep]
		\item The value of each state is the total amount of the reward an agent can expect to accumulate over the future, starting at that state.
		\item $v_\pi(s) = \mathbb{E}_\pi [\sum_{k=0}^\infty \gamma^k R_{t+k+1} | S_t = s]$. The Expected Reward given an State s.
		\item The agent takes the state with the biggest expected reward.
	\end{itemize} 
	\item \textbf{Policy Based}: optimize the policy function $a = \pi(s)$, without using the value function, a being the action to take, given a state.
	\begin{itemize}[noitemsep,nolistsep]
		\item The policy can either be deterministic, or stochastic $\pi(a|s) = \mathbb{P}[A=a|S=s]$(output is a distribution probebility over actions.)
		\item It directly indicates the best action to take for each step.
	\end{itemize} 
	\item \textbf{Model Based}: Model the environment. Each environment needs a different model foreach environment.
	\item Deep Reinforcement Learning: Uses deep neural networks to solve it.
\end{itemize}
\href{https://www.freecodecamp.org/news/diving-deeper-into-reinforcement-learning-with-q-learning-c18d0db58efe/}{Diving deeper into Reinforcement Learning with Q-Learning}
\begin{itemize}[noitemsep,nolistsep]
	\item \textbf{Q-learning} is value-based RL.
	\item \textbf{Q(Quality)-Table} gives you foreach action-state pair a value which moves gives the best maximum expected future reward.
	\item you don't implement a policy, you improve the Q-table to always choose the best action. The values in the table need to be learned.
	\item Action-Value Function (Q-Function) takes state and action as input and returns the expeced future reward.
	\item $Q^\pi(s_t,a_t) = \mathbb{E}\ [\sum_{k=0}^\infty \gamma^k R_{t+k+1} | s_t, a_t]$
	\item As we explore the environment, the Q-table will give us a better and better approximation by iteratively updating Q(s,a) using the \textbf{Bellman Equation}.
	\item Algorithm process: 1. Initialize Q-Table -> 2. Choose action a -> 3. perform action -> 4. measure reward -> 5. update Q -> goto 2.
	\begin{itemize}[noitemsep,nolistsep]
		\item 1. Initialize: e.g. initialize everything 0
		\item 2-3. choose an action. Use the epsilon greedy strategy. $ 0 \leq \epsilon \leq 1$ defines the exploration rate. It starts of with 1. We start of doing alot of random guesses what actions to choose (exploration). It is like a chance. We reduce the epsilon progressively to do more exploitation of the knowledge we gained.
		\item 4-5. update q: We update Q with the Bellman equation (given a new state s' and a reward r): $newQ(s,a) = Q(s,a) + \alpha[\Delta Q(s,a)],\ \Delta Q(s,a) = R(s,a) + \gamma \max(Q'(s',a')) - Q(s,a)$
		\item $\max(Q'(s',a'))$: Maxium expected future reward given the new s' and all possible actions at that new state. The highest Q-value between possible actions from the new state s'.
	\end{itemize} 
\end{itemize} 
\href{https://www.freecodecamp.org/news/an-introduction-to-deep-q-learning-lets-play-doom-54d02d8017d8/}{An introduction to Deep Q-Learning: let’s play Doom}
\begin{itemize}[noitemsep,nolistsep]
	\item Instead of using a \textbf{Q-table}, use a Neural Network that takes a state and \textbf{approximates Q-values} for each action based on that state.
	\item In a videogame states can be associated with frames. you need multiple state inputs (like 4).
	\item preprocessing is important to reduce the complexity of the states to reduce the computation time needed for training.
	\item \textbf{temporal limitation}: you need multiple frames to percept motion in the environment.
	\item using convolutional layers with ELU. Use fully connected layers with ELU and one output layer that produces the Q-value estimataion for each action.
	\item Making more efficient use of observed experience using experience Replay:
	\begin{itemize}[noitemsep,nolistsep]
		\item \textbf{Avoid forgetting previous experiences}: given that we use sequential samples from interactions with our environment, the network tends to forget the previous experiences. You could use previous experiences by learning it multiple times.
		\item reducing correlation between experiences: every action affects the next state, the sequence of experiences can be highly correlated. If we train in sequential order we might risk the agent bein influenced by it. Two strategies:
		\item stop learning while interacting with the environment. Play a little randomly to explore the state space. Then recall these experiences and learn from then, then play again with the updated value function.
		\item This way you have better set of examples. This prevents reinforcing the same action over and over.
	\end{itemize} 
	\item $\Delta w = \alpha [(R + \gamma\ max_a \hat{Q}(s',a,w)) - \hat{Q}(s,a,w)]\bigtriangledown_w \hat{Q}(s,a,w)$ 
	\item $\Delta w = \alpha * TD-Error * Gradient\ of\ our\ Prediction$
\end{itemize} 
\href{https://www.freecodecamp.org/news/improvements-in-deep-q-learning-dueling-double-dqn-prioritized-experience-replay-and-fixed-58b130cc5682/}{Improvements in Deep Q Learning: Dueling Double DQN, Prioritized Experience Replay, and fixed Q-targets}
\begin{itemize}[noitemsep,nolistsep]
	\item Fixed Q-targets:
	\begin{itemize}[noitemsep,nolistsep]
		\item We calculate \textbf{TD-Error} (aka the loss), but we don't have any idea of the real TD-target. Bellman equation states that the TD-Target is the reward of taking that action at that state plus the discounted highest Q-value for the next state.
		\item But we use the weights for the target and the Q-value and therefore our Q-value and our target value shifts.
		\item \textbf{Q-Targets}: Using a seperate network with a fixed parameter (w-tilde) for estimating the TD-Target. At every tau step, we copy the parameters from our DQN network to update the target network:
		\item $\Delta w = \alpha [(R + \gamma\ max_a \hat{Q}(s',a,\tilde{w})) - \hat{Q}(s,a,w)]\bigtriangledown_w \hat{Q}(s,a,w),\ At\ every \tau step: \tilde{w} \leftarrow w$
	\end{itemize} 
	\item \textbf{Double DQN}: Handles the problem of the overestimation of Q-values.
	\begin{itemize}[noitemsep,nolistsep]
		\item TD-Target = Q-target = reward + discounted max-q.
		\item How are we sure the best action for the next state ist the action with the highest Q-value, it depends on what actions we tried and what neighbors we explored.
		\item In the beginning of the training the max-q value will obviously b noisy and can lead to false positives. Learning will be complicated.
		\item Solution: When computing q-target, use two networks to decouple the action selection from the target Q-value generation
		\item Use our DQN network to select what is the best action to take for the next state (the action with the highest Q-value). We use our target network to calculate the target Q-value of taking that action at the next state.
		\item $argmax_a Q(s',a) = DQN\ choose\ action\ for\ next\ state,\ Q(s',argmax_a Q(s',a)) = Target\ network\ calculates\ the\ qvalue.$
		\item $Q(s,a) = r(s,a) + \gamma Q(s',argmax_a Q(s',a))$
		\item this helps us reduce the overestimation of q values and helps us train faster and have more stable learning.
	\end{itemize} 
	\item \textbf{Dueling DQN (aka DDQN)}: Seperate the estimator into two parts:
	\begin{itemize}[noitemsep,nolistsep]
		\item Q(s,a) can be decomposed as the sum of: V(s): the value of being at that state. A(s,a): the advantage of taking that action at that state (how much better it is to all other actions).
		\item With DDQN, we seperate the estimator using two streams one for V(s) and one for A(s,a) and then combine these two streams through a special aggregation layer to get an estimate of Q(s,a). Two streams in the NN.
		\item By decoupling the estimation we can learn which states are valuable without having to learn the effect of each action at each state.
		\item Being able to calculate V(s) can be useful for state where their actions do not affect the environment in a relevant way.
		\item Aggregation: Simply adding both streams will be problemantic for the back propagation, you can force the advantage function estimator to have 0 advantag at the chosen action. To do that, we subtract the average advantage of all actions possible of the state.
		\item $Q(s,a;\theta,\alpha,\beta) = V(s;\theta,\beta) + (A(s,a;\theta,\alpha) - \frac{1}{\mathcal{A}}\sum_a' A(s,a';\theta,\alpha))$
		\item $\theta: common\ network\ parameters,\ \alpha: advantage\ stream\ parameters,\ \beta: value\ stream\ parameters,\ the\ sum\ is\ the\ average.$
		\item This helps us accelerate the training. This helps us find much more relaible Q-values for each action by decoupling the estimation between two streams.
	\end{itemize} 
	\item \textbf{Prioritized Experience Replay}: Some experiences may be more important than others for our training, but might occur less frequently.
	\begin{itemize}[noitemsep,nolistsep]
		\item If we sample the experiences randomly these rich experiences that occur rarely have practilly no chance to be selected.
		\item Use a priority. where there is a big difference between our prediction and the TD target, since it means that we have a lot to learn about it.
		\item We use the absolute value of the magnitude of our TD-error: $p_t = |\delta_t| + e$,\ e = const, that assures that no experience has no 0 probability.
		\item Put that priority in the experience of each replay buffer to select the experiences.
		\item Do not go greedy prioritization: overfitting!. Stochastic prioritization: $P(i) = \frac{p_i^a}{\sum_k p_k^a}$,\ a reintroduces some randomness, a = 0 pure uniform randomness, a = 1 only select the experiences with the highest priorities.
		\item To combat over-fitting by prioritization of high-priority samples use Importance sampling weights (IS): $(\frac{1}{N} * \frac{1}{P(i)})^b$,\  b = controls how much the w affects learning. Close to 0 at the beginning of learning and annealed up to 1 over the duration of training. Because these weights are more important in the end of learning when our q-values begin to converge.
		\item To sort the replays use an unsorted sumtree
	\end{itemize} 
\end{itemize} 
\href{https://www.freecodecamp.org/news/an-introduction-to-policy-gradients-with-cartpole-and-doom-495b5ef2207f/}{An introduction to Policy Gradients with Cartpole and Doom}
\begin{itemize}[noitemsep,nolistsep]
	\item in policy-based methods we directly learn the policy function that maps state to action. we directly parameterize $\pi$
	\item Deterministic policies are used in deterministic environments. stochastic policy is used when the environment is uncertain. We call this process a Partially Observable Markov Decision Process (POMDP).
	\item \textbf{Advantage of Policy Gradients}:
	\begin{itemize}[noitemsep,nolistsep]
		\item \textbf{convergence}: policy-based methods have better convergence properties. value-based methods might oscillate alot. Policy based methods follow gradients we converge on a local maximum (worst case), or global maximum (best case).
		\item Policy gradient are more effective in \textbf{high dimensional action spaces}: as Deep Q-learning is that their prediction assign a score for each eaction at each time step, given the current state.
		\item Policy gradients \textbf{can learn stochastic policies}: value functions can't. In Policy we donÄt need to implement an exploration/explotation trade off. 
	\end{itemize}
	\item \textbf{Disadvantages of Policy Gradients}:
	\begin{itemize}[noitemsep,nolistsep]
		\item Alot of the time, they converge on a \textbf{local maximum} rather than on the global optimum.
		\item \textbf{Slower convergence}: Then Deep Q-Learning.
	\end{itemize}
	\item \textbf{Policy Search}: We have our policy $\pi$ that has a parameter $\theta$. THis pi outputs a probability distribution of actions.
	\begin{itemize}[noitemsep,nolistsep]
		\item $\pi_\theta (a|s) = P[a|s]$
		\item Good policy: theta that maximizes the score function: $J(\theta) = E_{\pi \theta} [\sum \gamma r]$
		\item \textbf{Steps}: 1st: Measure the quality of policy with a policy score function, 2nd: use policy gradient ascent to find best parameter theta that improves our policy.
		\item \textbf{1st Step}: The Policy Score function J(theta):
		\begin{itemize}[noitemsep,nolistsep]
			\item Episodic environment: Calculate the mean of the return from the first time Step (G1): $J_1(\theta) = E_\pi[G_1 = \sum_{k=0}^\infty \gamma^k R_{1 + k}] = E_\pi (V(s_1))$. We want a policy that optimizes G1, as this will be the best policy.
			\item Continious Environment: We can use the average value, because we can't rely on a specific start state and their values are now weighted by the probability of the occurrence of the respected state: $J_{avgv}(\theta) = E_\pi(V(s)) = \sum d(s)V(s),\ where\ d(s) = \frac{N(s)}{\sum_s'N(s')}$
			\item N(s) = Number of occurrences of the state.
			\item use the average reward per timestap: $J_{avR}(\theta) = E_\pi(r) = \sum_s d(s) \sum_a \pi_\theta(s,a) R_s^a$. sum over a: Probability that I take this action a from that state under this policy, Rsa: immediate reward that I get.
		\end{itemize}
		\item \textbf{2nd Step}: Policy gradient ascent.
		\begin{itemize}[noitemsep,nolistsep]
			\item To maximize the score function J(theta), we need to do gradient ascent on policy parameters.
			\item We use gradient ascent as the score function is not an error function (there we would use gradient descent.)
			\item Goal: $\theta^* = \underset{\theta}{argmax}E_{\pi \theta}[\sum_t R(s_t,a_t)]$, Score function: $J(\theta) = E_\pi[R(\tau)]$
			\item Problem: How do we estimate the Gradient with respect to theta, when the gradient depends on the unknown effect of policy changes on the state distribution?
			\item Solution: $\bigtriangledown_\theta J(\theta) = E_\pi [\bigtriangledown_\theta(log \pi (\tau|\theta))R(\tau)], \pi (\tau|\theta): policy\ function, R(\tau): score\ function$
			\item Update Rule: $\Delta \theta = \alpha * \bigtriangledown_\theta(log \pi (s, a, \theta))R(\tau)$
			\item R(tau): High value: it means thatn on average we took actions that lead to high rewards. If it is low, we want to push down the probabilities of the actions seen.
		\end{itemize}
		\item Policy gradient can be improved with Proximal Policy Gradients (ensure that the deviations from the previous policy stays relatively small) and Actor Critic (a hybrid between value-based algorithms and policy-based algorithms).
	\end{itemize}
\end{itemize} 
\href{https://www.freecodecamp.org/news/an-intro-to-advantage-actor-critic-methods-lets-play-sonic-the-hedgehog-86d6240171d/}{An intro to Advantage Actor Critic methods: let’s play Sonic the Hedgehog!}
\begin{itemize}[noitemsep,nolistsep]
	\item \textbf{Actor Critic}: Hybrid method. Use two neural networks: A Critic that measures how good the action takesn is (value-based) and an Actor that controls how our agent behaves (policy-based).
	\item State of the art: \textbf{Proximal Policy Optimization (PPO)}, is based on Advantage Actor Critic.
	\item \textbf{Policy Gradient Problem}: Reward is done for-each episode, so small bad decisions will be averaged out. And we won't find an optimal policy.
	\item Use TD-Learning: $\Delta \theta = \alpha * \bigtriangledown_\theta * (log \pi(S_t,A_t,\theta)) * Q(S_t,A_t)$. We do update each step sou we don't use the total rewards R(t). The Critic model approximates the value function.
	\item The critic will help to find the policy and update their own way to provide better feedback.
	\item Actor: $\pi(s,a,\theta)$  Critic: $\hat{q}(s,a,w)$
	\item Weights: Policy: $\Delta \theta = \alpha \bigtriangledown_\theta(log \pi_\theta(s,a)) * \hat{q}_w(s,a)$, Value: $\Delta w = \beta (R(s,a) + \gamma \hat{q}_w(s_{t+1},a_{t+1}) - \hat{q}_w((s_t,a_t))\ \bigtriangledown_w \hat{q}_w(s_t,a_t)$
	\item \textbf{Process}: At each time-step: current State St into Actor and Critic. Policy outputs Action At and receives a new State and a reward.
	\item The Critic computes the value of taking that action at that state and the actor updates is policy parameters (weights) using this q-value.
	\item To reduce the Variability: Use Advantage function: $A(s,a) = Q(s,a) - V(s)$ Q(s,a): q-value for action a in state s, V(s): average value of that state.
	\item This function calculates the extra reward I get if I take this action. A(s,a) > 0:  our gradient is pushed in that direction, A(s,a) < 0: our gradient is pushed in the opposite direction.
	\item Use the TD-Error as an good estimator: $A(s,a) = r + \gamma V(s') - V(s)$
	\item Strategies: Synchronous: \textbf{A2C} (Advantage Actor Critic), Asynchronous: \textbf{A3C} (Asynchronous Advantage Actor Critic).
	\item A3C uses different agents in parallel on multiple instances of the environment. Each worker will update the global network asynchronously.
	\item Problem of A3C: \href{https://lilianweng.github.io/lil-log/2018/04/08/policy-gradient-algorithms.html#a2c}{Link}. Because of asynchronous nature of A3C, some workers will be playing with older version of the parameters, thus the aggregating update will not be optimal. In A2C it waits for each actor to finish before updating the global parameters. Therefore the training will be more cohesive and faster.
	\item Each worker in A2C will ahve the same set of weights since, contrary to A3C, A2C updates all their workers at the same time. YOu can create multiple versions of environments and then execute them in parallel.
\end{itemize}
\href{https://towardsdatascience.com/proximal-policy-optimization-ppo-with-sonic-the-hedgehog-2-and-3-c9c21dbed5e}{Proximal Policy Optimization (PPO) with Sonic the Hedgehog 2 and 3}

\subsection{Deep RL Bootcamp}
\href{https://sites.google.com/view/deep-rl-bootcamp/lectures}{Deep RL Bootcamp}

\subsection{RL Lectures from Deepmind}
\href{https://www.youtube.com/watch?v=2pWv7GOvuf0&list=PLqYmG7hTraZDM-OYHWgPebj2MfCFzFObQ}{RL Course by DeepMind}
\\
\href{https://www.youtube.com/watch?v=2pWv7GOvuf0&list=PLqYmG7hTraZDM-OYHWgPebj2MfCFzFObQ}{RL Course by DeepMind - Part 1}
\begin{itemize}[noitemsep,nolistsep]
	\item Actions may have long term consequences and rewards may be delayed. May need to sacrifice immediate reward to gain more long-term reward.
	\item $Observation\ O_t, Reward\ R_t, Action\ A_t, History\ H_t\ (sequence\ of\ O_t,\ A_t,\ R_t)$
	\item $State\ S_t$ (simpler information to determine what happens next, usually function of history: $S_t = f(H_t)$
	\item State Definitions:
	\begin{itemize}[noitemsep,nolistsep]
		\item enviroment state $S_t^e$ is the enviroments private representation. Environment state not visible to the agent.
		\item agent state $S_t^a$ is the agents internal representation. Used to pick next action. $S_t^a = f(H_t)$
		\item markov state $A\ state\ S_t\ is\ Markov\ iff:\ \mathbb{P}[S_{t+1} | S_t] = \mathbb{P}[S_{t+1}|S_1,...,S_t]$ . You only need the current state to infer the next state or the future. A helicopter state needs velocity. Otherwise you need the complete history to calculate velocity if it only stored position.
		\item environment state $S_t^e$ and the history $H_t$ is Markov.
	\end{itemize} 
	\item Environments:
	\begin{itemize}[noitemsep,nolistsep]
		\item fully observability: agent directly observes environment state $O_t = S_t^a = S_t^e$. This is a Markov decision process (MDP).
		\item partial observability: $S_t^a \neq S_t^e$. This is a partially observable Markov decision process (POMDP). Agent constructs it's own $S_t^a$.
		\item partial observability state: complete history $S_t^a = H_t$, beliefs: $S_t^a = (\mathbf{P}[S_t^e = s^1],...,\mathbf{P}[S_t^e = s^n])$, recurrent NN: $S_t^a = \sigma(S_{t-1}^a W_s + O_t W_o)$ (linear transformation)
	\end{itemize} 
	\item Inside an RL Agent
	\begin{itemize}[noitemsep,nolistsep]
		\item policy (agent's behavior), value function (how good is state-action pair), model (agents representation of the environment).
		\item model: predicts what the environment will do next. you don't need to do models.
		\item Transitions: $\mathcal{P}$ predicts next state (dynamics). Rewards $\mathcal{R}$ predicts next immediate reward
		\item e.g.: $\mathcal{P}_{ss'}^a = \mathbf{P}[S=s'| S=s, A=a],\ \mathcal{R}_s^a = \mathbf{E}[R|S=s, A=a]$
		\item model-free agent: Policy and/or Value Function and no model.
		\item model-based agent: Policy and/or Value Function and a model. first build the dynamics of the environmant es with the model
	\end{itemize} 
	\item Problems with RL
	\begin{itemize}[noitemsep,nolistsep]
		\item RL-Problem: Environment initially unknown and the agent learns by interaction.
		\item Planning-Problem: Environment-model is known from the start. 
		\item Prediction: evaluate the future (given a policy) vs. Control: optimise the future (find the pest policy)
	\end{itemize} 
\end{itemize} 
\\
\href{https://www.youtube.com/watch?v=lfHX2hHRMVQ&list=PLqYmG7hTraZDM-OYHWgPebj2MfCFzFObQ&index=2}{RL Course by DeepMind - Part 2}


\section{ALR-Wiki}
\href{https://www.notion.so/ALR-Student-Wiki-284ed2bb033f4ef494e2429ce0f51b6c}{ALR-Wiki Link}
\\
\href{https://www.notion.so/Books-e1b1a91cf80c426da3eddd9fec5bdd38}{ALR-Wiki Books Link}
\\
\href{https://www.notion.so/Implementation-Guidelines-0afdb8f58c714f42b3526eb415194c8c}{Implementation Guidelines}
\\
\href{https://www.notion.so/ebd32c68f9cd4502a02af10e82e410cd?v=0eab511a5007451baa2c199f0d6c7d35}{Latex Templates}
\\
\href{https://www.notion.so/Writing-Checklist-56979e6a0227447f8638d30bcbc33362}{Writing Checklist}
\\
\href{https://www.notion.so/Talks-How-to-research-e9b534f4dc09418e82cac8d8e2da4f4d}{How to do Research}

\section{Books}
\subsection{Deep Learning Book}
\href{https://www.deeplearningbook.org/}{Deep Learning Book}
\\


\subsection{Reinforcement Learning - An Introduction} 
\href{http://incompleteideas.net/book/the-book.html}{Reinforcement Learning}
\\
a

\section{UNSORTED}
Gordon 2000: Ants at Work.
\\
Gordon 2007: Control without hierarchy. Nature.
\\\textbf{Links:}\\
\\
\href{https://www.youtube.com/watch?v=X-iSQQgOd1A}{Ant Simulation Video 1}
\\
\href{https://www.youtube.com/watch?v=81GQNPJip2Y}{Ant Simulation Video 2}
\\
\href{https://www.youtube.com/watch?v=bqtqltqcQhw}{Boids Video}
\\
\href{https://en.wikipedia.org/wiki/Distributed_artificial_intelligence}{Distributed Artificial Intelligence, Wikipedia}
\\
\href{https://en.wikipedia.org/wiki/Multi-agent_learning}{Multi-agent learning, Wikipedia}
\\
\href{https://en.wikipedia.org/wiki/Bees_algorithm}{Bees algorithm, Wikipedia}
\\
\href{https://en.wikipedia.org/wiki/Swarm_intelligence}{Swarm Intelligence, Wikipedia}


\section{References \& Papers}
\subsection{Ant Colony Optimization (ACO)}
\href{https://www.sciencedirect.com/science/article/pii/S0888613X02000919}{ACO - Ant Colony Optimization for learning Bayesian network - 2002}

\subsection{Graph Neural Networks}
\underline{(Base)}
\href{https://www.youtube.com/watch?v=uF53xsT7mjc}{GNN - Sur - Theoretical Foundations of Graph Neural Networks - 2021}
\\
\begin{itemize}[noitemsep,nolistsep]
	\item Goal: Exact same results for two \textbf{isomorphic graphs} (graphs that are the same the nodes are just arranged differently).
	\item Nodes: $x_i \in \mathbb{R}^k$ (features of node), feature matrix $\mathbf{X} = (x_1,...x_n)^T \in \mathbb{R}^{k \times n}$
	\item By stacking the nodes in the matrix you have already ordered them (result should not depend on this).
	\item Operations that change the nod order: permutation matrices. They ahve exactly one 1 in every row and column, and zeroes everywhere examples. Left Multiplied: permute the rows. $P_{(2,4,1,3)}$: The numbers indicate where the 1 in the row is.
	\item \textbf{Permutation Invariance}: f is permutation invariant iff: $\forall P \in Permutation: f(PX) = f(X)$. Example: Deep Sets Model $f(X) = \phi(\sum_{i \in V} \psi(x_i))$. This is true for the entire data-set.
	\item \textbf{Permutation equivariance}: for identification on the node level. Seek functions that don't change the node order. f is permutation equivariant iff: $\forall P \in Permutation: f(PX) = Pf(X)$.
	\item \textbf{equivariance}: each node's row is unchanged by f. So foreach node we could define: $\forall i: h_i = \psi(x_i).$, the latent features h-i. Stacking h yields: $H=f(x)$. The functions are applied indipendently to each node.
	\item Stacking equivarent functions with an invariant tail: $f(X) = \phi(\bigoplus_{i \in V}\psi(x_i)). \bigoplus\ is\ permutationinvariant\ aggregator\ (sum,\ avg,\ max)$.
	\item \textbf{Learning on Graphs}:
	\begin{itemize}[noitemsep,nolistsep]
		\item Represent Edges with adjacency matrix A: $a_{ij} = \begin{cases} 1& (i,j) \in E \\ 0 & otherwise \end{cases}$. Edge features could be added aswell. permutation x-variance still holds.
		\item xvariance on graphs: To not change edges: permutate rows and columns. Permutate with $PAP^T$.
		\item \textbf{Invariance}: $f(PX,PAP^T) = f(X,A)\ (A = Edges,\ X = Nodes)$
		\item \textbf{Exvariance}: $f(PX,PAP^T) = Pf(X,A)\ (A = Edges,\ X = Nodes)$
		\item Neighbourhoods: Node i, it's 1-hop neighbors are defined as: $\mathcal{N}_i = \{j: (i,j) \in E \vee (j,i) \in E\}$. (Non-directed edges, node i is in it's own neighbourhood).
		\item Multiset of features in the neighbourhood: $X_{\mathcal{N}_i} = \{\{ x_j : j \in \mathcal{N}_i \}\}$. With a local function g as operating over this multiset: $g(x_i, X_{N_i})$
		\item Construct perm-equi function f(X,A) by applying g over all neighbourhoods: $f(\textbf{X,A}) = \begin{pmatrix} g(x_1,X_{\mathcal{N}_1}) \\ g(x_2,X_{\mathcal{N}_1}) \\ \vdots \\ g(x_n,X_{\mathcal{N}_n}) \end{pmatrix}$. g should not depend on the order of the neighbourhood, it should be permu-invari.
		\item Once you have the latent-Graph via the GNN you can use them in a Node-classification,  Graph-classification, or Link-prediction task.
	\end{itemize}
	\item Message Passing in Graphs.
	\begin{itemize}[noitemsep,nolistsep]
		\item GNN Layer: Construct f(X,A) via the local function g (known as diffusion, propagation or message passing). F is refered ta as a GNN layer.
		\item How to define g? Active research!
		\item Classification thre flavoulrs of CNN:
		\item Convolutional GNN:
		\begin{itemize}[noitemsep,nolistsep]
			\item constants $c_{ij}$. How much does Node i value the features of nodes j. They are coefficients for weighted combinations. The weights usually depend on A.
			\item $h_i = \phi (x_i, \bigoplus_{j \in \mathcal{N}_i} c_{i,j}\psi(x_j))$. 
			\item Examples: ChebyNet, GCN (Graph Convolutional Network), SGC (Simplified Graph Convolutional Networks)
			\item useful for homophilous graphs (similar edges) and scales well.
		\end{itemize}
		\item Attentional GNN:
		\begin{itemize}[noitemsep,nolistsep]
			\item neighbour features aggregated with implicit weights (via attention a). This weights are learnable.
			\item $h_i = \phi (x_i, \bigoplus_{j \in \mathcal{N}_i} a(x_i,x_j)\psi(x_j))$. 
			\item Examples: MoNet, GAT (Graph Attention Network), GaAN (Gated Attention Network).
			\item useful as a middle ground with respect to capacity and scale. Edges are not strict homophily, but you compute sclarar value in each edge.
		\end{itemize}
		\item Message Passing GNN:
		\begin{itemize}[noitemsep,nolistsep]
			\item sender and receiver work together to compute arbitrary vectors ("messages") to be sent across edges.
			\item $h_i = \phi (x_i, \bigoplus_{j \in \mathcal{N}_i} \psi(x_i, x_j)). \psi(x_i, x_j)) = m_{ij}$.  
			\item Examples: Interaction Networks, MPNN (Message Passing Neural Networks), GraphNets
			\item most generic GNN. May have scalability or learnability issues. Ideal for reasoning.
		\end{itemize}
	\end{itemize}
	\item Node embedding techniques:
	\begin{itemize}[noitemsep,nolistsep]
		\item embedding: Finding an Encoding, so that x-i are now the latent features of h-i.
		\item a good representation should preserve the graph structure. This leads to the unsupervised objective: $optimise\ h_i\ and\ h_j\ to\ be\ nearby\ iff: (i,j) \in E$. They predict if there is an edge between the nodes.
		\item Can use binary cross-entropy loss: $\sum_{(i,j) \in E}log \sigma(h_i^Th_j) + \sum_{(i,j) \notin E}log (1 - \sigma(h_i^Th_j))$
	\end{itemize}
	\item local objective emulate Convolutional GNNs. Neighbouring nodes tend to highly overlap in n-step neighborhoods. A conv-GNN enforces similar features for neighbouring nodes by design.
	\item GNN and Natural Language Processing (NLP) correspond alot (nodes similar to words).
	\item Common assumption if you have no information about how the graph should look like: Assume a complete graph and then let the network infer the actual relations.
	\item Transformers: are fully connected attentional GNNs.
	\item Spectral GNNs:
	\begin{itemize}[noitemsep,nolistsep]
		\item Time Sequences can be imageind as a cyclical grid graph with a convolution over it. A node is a time-step and the convolution looks at the time step and it's immediate neighbors.
		\item You don't need to know the convolutional operation if you know the eigenvalues with respect to the fourier basis (36:13)
		\item convolutional GNN: $c_{ij} = (p_k(L))_{ij}$. Use a polynomial function p-k for the Laplacian matrix L = D - A. D being the Degree matrix. p-k is necessary to make the eigenvalue decomposition easier.
		\item This means there is no spectral GNN and spatial GNN as they can be transformed into each other.
	\end{itemize}
	\item Probabilistic Graphical Models:
	\begin{itemize}[noitemsep,nolistsep]
		\item Nodes are random variables and edges are dependencies between their distributions. THis is an Probabilistic graphical Model (PGMs). This helps you decompose a joint probability distribution.
		\item Can use Markov Random Fields (MRF) to decompose the joint into a product of edge potentials.
		\item Mean-field inference.
		\item PGM corresponds to a message passing GNN.
	\end{itemize}
	\item how powerful are GNNs?
	\begin{itemize}[noitemsep,nolistsep]
		\item untrained GNNs work well, as they are similar to random hashes. (Weisfeiler-Lemann Test). Also called 1-WL test.
		\item Though some instances the isomorphism test fails.
		\item GNNs can only be as powerful as the 1-WL test.
		\item Can make the strongers by analysing failer cases.
		\item Continous Features: Sums may not destinguish parts of the graph (2+2 = 4+0).
	\end{itemize}
	\item \href{https://youtu.be/uF53xsT7mjc?t=2943}{curr}
\end{itemize}

\subsection{Reinforcement Learning}
\href{https://medium.com/emergent-future/simple-reinforcement-learning-with-tensorflow-part-0-q-learning-with-tables-and-neural-networks-d195264329d0}{Multiple blog post for RL (even has A3C)}
\\
\href{https://www.nature.com/articles/nature14236}{RL - DQN - Human-level control through deep reinforcement - 2015}
\\
\href{http://proceedings.mlr.press/v48/mniha16.html}{RL - A3C - Asynchronous Methods for Deep Reinforcement Learning - 2016}
\\
\underline{(Sur)veys/Reviews}
\\
\href{https://www.researchgate.net/publication/338396174_State-of-the-Art_Reinforcement_Learning_Algorithms}{RL - Sur - State-of-the-art Reinforcement LEarning Algorithms - 2020}

\subsection{Multiagent Systems (MAS)}
\underline{(App)lications}
\\

\underline{Actor-Critic (AC)}
\\

\underline{(Base)}
\\
\href{https://www.hindawi.com/journals/complexity/2017/3813912/}{MAS - Base - The Multiagent Planning Problem - 2016}
\\

\underline{(Com)munication}
\\

\underline{(Con)ference}
\\
\href{https://link.springer.com/book/10.1007%2F978-981-33-6718-0}{MAS - Con - Distributed Cooperative Control and Communication for Multi-agent Systems - 2021}
\\
\href{https://link.springer.com/book/10.1007%2F978-3-030-69322-0}{MAS - Con - PRIMA 2020 Principles and Practice of Multi-Agent Systems - 2021}
\\
\href{https://link.springer.com/book/10.1007/978-3-642-15461-4}{MAS - Con - Swarm Intelligence - 2010}
\\
\href{https://link.springer.com/book/10.1007/978-3-642-32650-9}{MAS - Con - Swarm Intelligence - 2012}
\\
\href{https://link.springer.com/book/10.1007/978-3-319-09952-1}{MAS - Con - Swarm Intelligence - 2014}
\\
\href{https://link.springer.com/book/10.1007/978-3-319-44427-7}{MAS - Con - Swarm Intelligence - 2016}
\\
\href{https://link.springer.com/book/10.1007/978-3-030-00533-7}{MAS - Con - Swarm Intelligence - 2018}
\\
\href{https://link.springer.com/book/10.1007/978-3-030-60376-2}{MAS - Con - Swarm Intelligence - 2020}
\\

\underline{(Evo)lutionary}
\\
\href{https://link.springer.com/chapter/10.1007/978-3-540-71618-1_8}{MAS - Evo - Co-evolutionary Multi-agent System with Predator-Prey Mechanism for Multi-objective Optimization - 2007}
\\

\underline{(Het)erogeneous}
\\
\href{https://link.springer.com/article/10.1023/A:1008942012299}{MAS - Het - Multiagent Systems A Survey from a Machine Learning Perspective - 2000}
\\

\underline{(Hie)rarchy}
\\
\href{https://ieeexplore.ieee.org/abstract/document/4427756}{MAS - Hie - Hierarchical Control in a Multiagent System - 2007}
\\
\href{https://link.springer.com/chapter/10.1007/978-3-540-25928-2_6}{MAS - Hie - Holonic - A Taxonomy of Autonomy in Multiagent Organisation - 2003}
\\

\underline{Multi-Objective (MO)}
\\

\underline{(Role)-Based}
\\

\underline{(Sca)ling}
\\

\underline{(Sur)veys/Reviews}
\\
\href{https://ieeexplore.ieee.org/abstract/document/8352646}{MAS - Multi-Agent Systems - A Survey - 2018}
\\
\href{https://dl.acm.org/doi/abs/10.1017/S0269888905000317}{MAS - Sur - A survey of multi-agent organizational paradigms - 2004}
\\

\underline{(Tra)nsferlearning}
\\
\href{https://www.scitepress.org/Papers/2011/31856/}{MAS - Tra - Transfer Learning for Multi-agent Coordination - 2011}
\\
\href{https://ulir.ul.ie/handle/10344/3305}{MAS - Tra - Transfer learning in multi-agent systems through paralllel transfer - 2013}
\\


\subsection{Multi Agent Reinforcement Learning (MARL)}
\underline{(App)lications}
\\

\underline{Actor-Critic (AC)}
\\
\href{https://ieeexplore.ieee.org/abstract/document/8619581}{MARL - AC - Networked Multi-Agent Reinforcement Learning in Continuous Spaces - 2018}
\\
\href{https://arxiv.org/abs/1706.02275}{MARL - AC - Multi-Agent Actor-Critic for Mixed Cooperative-Competitive Environment - 2017}
\\
\href{http://proceedings.mlr.press/v97/iqbal19a.html}{MARL - AC - Actor-Attention-Critic for Multi-Agent Reinforcement Learning - 2019}
\\

\underline{(Base)}
\href{https://dl.acm.org/doi/abs/10.5555/645527.657296}{MARL - Base - Multiagent Reinforcement Learning - Theoretical Framework and an Algorithm - 1998}
\\
\href{https://publikationen.bibliothek.kit.edu/1000118251}{MARL - Base - Deep Reinforcement Learning for Robot Swarms - 2019 - KIT}
\\
\href{https://arxiv.org/abs/2009.14471}{MARL - Base - PettingZoo - Gym for Multi-Agent Reinforcement Learning - 2020}
\\
\href{https://www.sciencedirect.com/science/article/pii/S0893608099000246}{MARL - Base - Multi-agent reinforcement learning weighting and partitioning - 1999}
\\

\underline{(Com)munication}
\\
\href{https://arxiv.org/abs/1605.06676}{MARL - Com - Learning to Communicate with Deep Multi-Agent Reinforcement Learning - 2016}
\\
\href{https://dl.acm.org/doi/abs/10.5555/2484920.2485093}{MARL - Com - Coordinating multi-agent reinforcement learning with limited communication - 2013}
\\

\underline{(Con)ference}
\\

\underline{(Evo)lutionary}
\\

\underline{(Het)erogeneous}
\\
\href{https://proceedings.neurips.cc/paper/2019/hash/07a9d3fed4c5ea6b17e80258dee231fa-Abstract.html}{MARL - Het - LIIR - Learning Individual Intrinsic Reward inMulti-Agent Reinforcement Learning - 2019}
\\
\href{https://www.sciencedirect.com/science/article/abs/pii/S092188900300040X}{MARL - Het - An approach to the pursuit problem on a heterogeneous multiagent system using reinforcement learning - 2002}

\underline{(Hie)rarchy}
\\
\href{https://link.springer.com/article/10.1007/s10458-006-7035-4}{MARL - Hie - Hierarchical multi-agent reinforcement learning - 2006}
\\

\underline{Multi-Objective (MO)}
\\
\href{https://www.cambridge.org/core/journals/knowledge-engineering-review/article/reward-shaping-for-knowledgebased-multiobjective-multiagent-reinforcement-learning/75F1507F7CAC7C6625F87AE7CD344D52}{MARL - MO - Reward shaping for knowledge-based multi-objective multi-agent reinforcement learning - 2017}
\\

\underline{(Role)-Based}
\\
\href{http://proceedings.mlr.press/v119/wang20f.html}{MARL - Role - ROMA Multi-Agent Reinforcement Learning with Emergent Roles - 2020}
\\

\underline{(Sca)ling}
\\
\href{https://link.springer.com/article/10.1007/s10489-020-01755-8}{MARL - Sca - GAMA - Graph Attention Multi-agent reinforcement learning algorithm for cooperation - 2020}
\\
\href{https://www.cambridge.org/core/journals/knowledge-engineering-review/article/planbased-reward-shaping-for-multiagent-reinforcement-learning/B173D25B1006B755667616C3A3EB3BE5}{MARL - Sca - Plan-based reward shaping for multi-agent reinforcement learning - 2016}
\\
\href{https://www.mdpi.com/2073-8994/10/10/461}{MARL - Sca - Multi-Agent Reinforcement Learning Using Linear Fuzzy Model Applied to Cooperative Mobile Robots - 2018}
\\
\href{http://proceedings.mlr.press/v70/foerster17b.html}{MARL - Sca - Stabilising Experience Replay for Deep Multi-Agent Reinforcement Learning - 2017}
\\
\href{http://proceedings.mlr.press/v80/yang18d.html}{MARL - Sca - Mean Field Multi-Agent Reinforcement Learning - 2018}
\\
\href{https://link.springer.com/chapter/10.1007/3-540-62934-3_39}{MARL - Sca - A modular approach to multi-agent reinforcement learning - 2005}
\\

\underline{(Sur)veys/Reviews}
\\
\href{http://ai.stanford.edu/people/shoham/www%20papers/MALearning_ACriticalSurvey_2003_0516.pdf}{MARL - Sur - Multi-Agent Reinforcement Learning - a critical survey - 2003}
\\
\href{https://arxiv.org/abs/1911.10635}{MARL - Sur - Multi-Agent Reinforcement Learning A Selective Overview of Theories and Algorithms - 2021}
\\
\href{https://arxiv.org/abs/1807.09427}{MARL - Sur - Multi-Agent Reinforcement Learning A Report on Challenges and Approaches - 2018}
\\
\href{https://arxiv.org/abs/1908.03963}{MARL - Sur - A Review of Cooperative Multi-Agent Deep Reinforcement Learning - 2019}
\\
\href{https://www.jair.org/index.php/jair/article/view/11396}{MARL - Sur - A Survey on Transfer Learning for Multiagent Reinforcement Learning Systems  - 2019}
\\

\underline{(Tra)nsferlearning}
\\
\href{https://link.springer.com/chapter/10.1007/978-3-642-29946-9_25}{MARL - Tra - Transfer Learning in Multi-agent ReinforcementLearning Domains - 2011}
\\
\href{https://ieeexplore.ieee.org/abstract/document/8851784}{MARL - Tra - Parallel Transfer Learning in Multi-Agent Systems What, when and how to transfer - 2019}
\\
\href{https://arxiv.org/abs/2002.08030}{MARL - Tra - Transfer among Agents An Efficient Multiagent Transfer Learning Framework - 2020}
\\
\href{https://link.springer.com/article/10.1007/s10458-019-09430-0}{MARL - Tra - Agents teaching agents a survey on inter-agent transfer learning - 2019}
\\


\subsection{GNN for Multi Agent Reinforcement Learning (GNNMARL)}
% Keywords: Graph Convolutional Network (GCN)
\underline{(App)lications}
\href{https://arxiv.org/abs/2011.06175}{GNN - App - Optimizing Large-Scale Fleet Management on a Road Network using Multi-Agent Deep Reinforcement Learning with Graph Neural Network - 2020}
\\

\underline{UNSORTED!!!!!}
\href{https://arxiv.org/abs/1811.12557}{GNN - Deep Multi-Agent Reinforcement Learning with Relevance Graphs - 2018}
\\
\href{https://arxiv.org/abs/2006.11438}{GNN - Deep Implicit Coordination Graphs for Multi-agent Reinforcement Learning- 2020}
\\
\href{https://ojs.aaai.org/index.php/AAAI/article/view/6211}{GNN - Multi-Agent Game Abstraction via Graph Attention Neural Network  - 2020}
\\
\href{https://arxiv.org/abs/2003.01040}{GNN - Scaling Up Multiagent Reinforcement Learning for Robotic Systems Learn an Adaptive Sparse Communication Graph - 2020}
\\
\href{https://ieeexplore.ieee.org/abstract/document/9413716}{GNN - Graphcomm A Graph Neural Network Based Method for Multi-Agent Reinforcement Learning - 2021}
\\
\href{https://arxiv.org/abs/2009.13161}{GNN - Towards Heterogeneous Multi-Agent Reinforcement Learning with Graph Neural Networks - 2020}
\\
\href{https://arxiv.org/abs/2008.02616}{GNN - The Emergence of Adversarial Communication in Multi-Agent Reinforcement Learning - 2020}
\\
\href{https://arxiv.org/abs/2105.10211}{GNN - Multi-Agent Deep Reinforcement Learning using Attentive Graph Neural Architectures for Real-Time Strategy Games - 2021}
\\
\href{https://ieeexplore.ieee.org/abstract/document/9414993}{GNN - Global-Localized Agent Graph Convolution for Multi-Agent Reinforcement Learning - 2021}
\\
\href{https://arxiv.org/abs/2012.07617}{GNN - Specializing Inter-Agent Communication in Heterogeneous Multi-Agent Reinforcement Learning using Agent Class Information - 2020}
\\
\href{https://dl.acm.org/doi/abs/10.5555/1248547.1248612}{GNN - Collaborative Multiagent Reinforcement Learning by Payoff Propagation - 2006}
\\


\subsection{Applications}
\href{https://ieeexplore.ieee.org/abstract/document/4012019}{MAS - TrafficControl - Neural Networks for Continuous Online Learning and Control - 2006}
\\