%% LaTeX2e class for student theses
%% sections/abstract_en.tex
%% 
%% Karlsruhe Institute of Technology
%% Institute for Program Structures and Data Organization
%% Chair for Software Design and Quality (SDQ)
%%
%% Dr.-Ing. Erik Burger
%% burger@kit.edu
%%
%% Version 1.3.5, 2020-06-26

%goal: ~150-225 words
\Abstract
This thesis explores a multitude of approaches for crafting an AI for a real-time strategy video game, that uses learning systems to provide an adaptive system, which allows for a sophisticated player experience with little external input during production and after shipment. Given that RTS games construct a complex and rich simulation environment for single-agent and multi-agent systems, they allow advancements in AI research to be benchmarked virtually and flexibly in an efficient manner. It will become apparent, that alot of the challenges can be dissected into loosely-coupled systems, that can be solved individually by different AI and learning methods. The focus will be on the usage of multi-agent systems. 
\bigskip

First of all it is going into more depth about why this research is interesting for the perspective of AI-Research and video game production and especially in comparison with classical video game AI approaches. Then it will go through theoretical fundamentals that are needed to be understood for this thesis and talks about how you would dissect the AI into different parts. Afterwards it will go over what is needed to solve the specific problem for the included project with differend solutions and approaches, while focussing on multi-agent systems. The project will entail the implementation of a part of the aforementioned dissected construct with the Godot Engine and a simplified prototype of a RTS game. In closing, the thesis will go over what still needs to be done for a complete AI suite and what aspects can still be improved upon.

(related works chapter???)