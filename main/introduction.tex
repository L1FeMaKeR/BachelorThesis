% !TeX spellcheck = en_US
% !TeX encoding = UTF-8
% !TeX root = ../thesis.tex

\chapter{Introduction}
% pages: 0.9-1.2 ~ 369-715 words, ~62-120 words per paragraph
% 130 words
Multi-Agent Systems has several real-world applications. Any System that requires multiple agents to work together or against can be modeled in a Multi-Agent fashion. In \citet{MARLTraffic2020} the authors define a traffic control system in which they create a grid of traffic lights, where each of them is an agent. They then learn the agents to optimize minimal sum of queue and waiting times at the intersections. Robot swarms can also be used to solve pathfinding problems in indoor, maze-like scenarios as discussed in \citet{SwarmPathFinding2013}. Given the structure of the environment communication is severly limited. 
% 2 more real-world problems. (Biology, Ants, Bees?)
\par

Recent applications and research in MARL and GNN.
\begin{itemize}[noitemsep,nolistsep]
	\item GNN really hot right now! Alot of research in this topic.
	\item MARL: Robin Paper (this thesis is based on his!). 
	\item Multi-Agent RL: RTS Games: \citet{RTSMARL2021}.
	\item GNN: Graph Convolutional reinforcement learning.
\end{itemize} \par

What is GNN, What is MARL, what can GNNs it do for MARL? (the main thing we want to talk about, more conceptually)
\begin{itemize}[noitemsep,nolistsep]
	\item MARL: Multiple Agents are trying to learn in an environment using RL Methods. Adapting RL algorithms designed for single agent environments for MAS.
	\item GNN: 
	\item What can GNN do for MARL:
\end{itemize} \par

What is my approach I want to talk about here? What was our goal?
\begin{itemize}[noitemsep,nolistsep]
	\item Approach: Based on Robin Paper => Expand on his with GNN! (this is a bachelor thesis) \citet{RobinRuede2021}
	\item restriction on MARL, restriction about agents, environment etc. (Robin, 2.4)
	\item Goal: Show that MARL especially under harsh communication ranges can benefit alot from GNN structures with multiple hops.
\end{itemize} \par

My work relative to other work. What has other research focused on?
\begin{itemize}[noitemsep,nolistsep]
	\item 
\end{itemize} \par

The following thesis is structured like this: Firstly we will go over the necessary preliminaries needed to understand this thesis in \Cref{ch:Preliminaries}. This starts of talking about Reinforcement Learning basics in the single-agent case, then adresses how it is applied in multi-agent tasks and is followed by a brief overview of Graph Neural Networks. In \Cref{ch:Related Work} related work will be discussed. The subsequent \Cref{ch:Architecture} is focused on the details of our proposed model. It will describe our policy architecture with an inclusion of homogeneous and heterogeneous Graph Neural Networks. Our experiments, including the general setup and the specific tasks that we trained our mode on, are described in \Cref{ch:Experiments}. The interpretation and evaluation of our results can be found in \Cref{ch:Evaluation}. We will complete this thesis by distilling the conclusions about our findings and provide future avenues for improvements in \Cref{ch:Conclusion}.